\section{Файловые диалоги и работа с файлами}

\textbf{Задание:} Создать таблицу Work. В другой файл вывести данные о рабочих, занимающих данную должность. (Вариант 14)

Вид окна представлен на рисунке \ref{fig:task8_form}.
\begin{figure}[H]
    \centering
    \includegraphics[scale=0.7]{task8/form.png}
    \caption{Внешний вид формы программы}
    \label{fig:task8_form}
\end{figure}

У элементов изменены значения некоторых атрибутов. 
Значения измененных атрибутов представлены в таблице \ref{table:params8}.

\begin{longtable}{|l|l|}
    Наименование атрибута & Значение\cr\hline
    \multicolumn{2}{|l|}{Для формы}\cr\hline
    \verb"Text" & \verb"Браузер "Амиго""\cr\hline
    \verb"FormBorderStyle" & \verb"FixedSingle"\cr\hline
    \verb"MaximizeBox" & \verb"False"\cr\hline
    \multicolumn{2}{|l|}{Для таблицы}\cr\hline
    \verb"(Name)" & \verb"gridResult"\cr\hline
    \multicolumn{2}{|l|}{Для кнопки записи в файл}\cr\hline
    \verb"(Name)" & \verb"SaveFileBtn"\cr\hline
    \multicolumn{2}{|l|}{Для кнопки открытия файла}\cr\hline
    \verb"(Name)" & \verb"OpenFileBtn"\cr\hline
    \multicolumn{2}{|l|}{Для кнопки сохранения результата}\cr\hline
    \verb"(Name)" & \verb"ResultBtn"\cr\hline
    \multicolumn{2}{|l|}{Для текстового поля должности}\cr\hline
    \verb"(Name)" & \verb"ResultTextBox"\cr\hline
    \multicolumn{2}{|l|}{Для обработчика ошибок}\cr\hline
    \verb"(Name)" & \verb"errorProvider1"\cr\hline

    \caption{Значения атрибутов элементов в приложении <<Работа с файлами>>}
    \label{table:params8}
\end{longtable}

Кроме того, приложение содержит элементы \verb|openFileDialogue|
и \verb|saveFileDialogue|, реализующие открытие и сохранение файлов.

Работа с ними производится в кнопках. Ниже приведен пример работы
функции открытия файла:

\begin{minted}[fontsize=\small, breaklines=true, style=bw, linenos]{cpp}
    private: System::Void OpenFileBtn_Click(System::Object^ sender, System::EventArgs^ e) {
			System::IO::Stream^ myStream;
			if (this->openFileDialog->ShowDialog() == System::Windows::Forms::DialogResult::OK) {
				CreateEmptyMatrix(gridResult, 1);
				if ((myStream = openFileDialog->OpenFile()) != nullptr) {
					System::IO::StreamReader^ sw = gcnew System::IO::StreamReader(myStream, System::Text::Encoding::
						GetEncoding(65001)); // UTF-8
					System::String^ s = "";
					int innerIdx = 0;
					while ((s = sw->ReadLine()) != nullptr && s != "") {
						gridResult->Rows->Add(1);
						int idx = 0;
						int current = 0;
						while (idx != s->Length) { 
							System::String^ currentWord = "";
							while (idx < s->Length && s[idx] != ' ') {
								currentWord += s[idx++];
							}
							if (idx < s->Length && s[idx] == ' ') ++idx;
							gridResult->Rows[innerIdx]->Cells[current++]->Value = currentWord;
							currentWord = "";
						}
						++innerIdx;
					}
					sw->Close();
				}
			}
		}
\end{minted}

После запуска программы на экране появляется окно следующего вида:
\begin{figure}[H]
    \centering
    \includegraphics[scale=0.4]{task8/exec.png}
    \caption{Внешний вид окна приложения}
\end{figure}

После открытия файла состояние программы изменится на следующее:
\begin{figure}[H]
    \centering
    \includegraphics[scale=0.4]{task8/openFile.png}
    \caption{Состояние программы после открытия файла}
\end{figure}

Результатом работы программы является новый файл:
\begin{figure}[H]
    \centering
    \includegraphics[scale=0.4]{task8/result.png}
    \caption{Результат работы программы}
\end{figure}

Полный код программы приведен в приложении А.


\documentclass[bachelor, och, pract]{SCWorks}
% параметр - тип обучения - одно из значений:
%    spec     - специальность
%    bachelor - бакалавриат (по умолчанию)
%    master   - магистратура
% параметр - форма обучения - одно из значений:
%    och   - очное (по умолчанию)
%    zaoch - заочное
% параметр - тип работы - одно из значений:
%    referat    - реферат
%    coursework - курсовая работа (по умолчанию)
%    diploma    - дипломная работа
%    pract      - отчет по практике
%    pract      - отчет о научно-исследовательской работе
%    autoref    - автореферат выпускной работы
%    assignment - задание на выпускную квалификационную работу
%    review     - отзыв руководителя
%    critique   - рецензия на выпускную работу
% параметр - включение шрифта
%    times    - включение шрифта Times New Roman (если установлен)
%               по умолчанию выключен
\usepackage[T2A]{fontenc}
\usepackage[utf8]{inputenc}
\usepackage{graphicx}

\usepackage[sort,compress]{cite}
\usepackage{amsmath}
\usepackage{amssymb}
\usepackage{amsthm}
\usepackage{fancyvrb}
\usepackage{longtable}
\usepackage{array}
\usepackage[english,russian]{babel}
\usepackage{minted}
\usemintedstyle{xcode}
\usepackage{tempora}
\usepackage{tabularx}


\usepackage[colorlinks=false]{hyperref}


\newcommand{\eqdef}{\stackrel {\rm def}{=}}

\newtheorem{lem}{Лемма}

% % При использовании biblatex вместо bibtex
%\usepackage[style=gost-numeric]{biblatex}
%\addbibresource{thesis.bib}

\begin{document}

% Кафедра (в родительном падеже)
\chair{математической кибернетики и компьютерных наук}

% Тема работы
\title{«Разработка приложений Windows.Forms на языке C++ в
среде Microsoft Visual Studio}

% Курс
\course{2}

% Группа
\group{251}

% Факультет (в родительном падеже) (по умолчанию "факультета КНиИТ")
%\department{факультета КНиИТ}

% Специальность/направление код - наименование
%\napravlenie{02.03.02 "--- Фундаментальная информатика и информационные технологии}
%\napravlenie{02.03.01 "--- Математическое обеспечение и администрирование информационных систем}
%\napravlenie{09.03.01 "--- Информатика и вычислительная техника}
\napravlenie{09.03.04 "--- Программная инженерия}
%\napravlenie{10.05.01 "--- Компьютерная безопасность}

% Для студентки. Для работы студента следующая команда не нужна.
%\studenttitle{Студентки}

% Фамилия, имя, отчество в родительном падеже
\author{Рыданова Никиты Сергеевича}

% Заведующий кафедрой
\chtitle{доцент, к.\,ф.-м.\,н.} % степень, звание
\chname{С.\,В.\,Миронов}

%Научный руководитель (для реферата преподаватель проверяющий работу)
\satitle{доцент, к.\,ф.-м.\,н.} %должность, степень, звание
\saname{А.\,.С\,Иванова}

% Руководитель практики от организации (только для практики,
% для остальных типов работ не используется)
\patitle{доцент, к.\,ф.-м.\,н.}
\paname{А.\,.С\,Иванова}

% Семестр (только для практики, для остальных
% типов работ не используется)
\term{1}

% Наименование практики (только для практики, для остальных
% типов работ не используется)
\practtype{учебная}

% Продолжительность практики (количество недель) (только для практики,
% для остальных типов работ не используется)
\duration{20}

% Даты начала и окончания практики (только для практики, для остальных
% типов работ не используется)
\practStart{01.09.20}
\practFinish{12.01.21}

% Год выполнения отчета
\date{2021}

\maketitle

% Включение нумерации рисунков, формул и таблиц по разделам
% (по умолчанию - нумерация сквозная)
% (допускается оба вида нумерации)
%\secNumbering


\tableofcontents

% Раздел "Обозначения и сокращения". Может отсутствовать в работе

% Раздел "Определения". Может отсутствовать в работе
%\definitions

% Раздел "Определения, обозначения и сокращения". Может отсутствовать в работе.
% Если присутствует, то заменяет собой разделы "Обозначения и сокращения" и "Определения"
%\defabbr


% Раздел "Введение"

\intro

Целью практики является освоение механизма построения оконного интерфейса приложений в среде Visual Studio.
В результате прохождения практики должны быть отработаны навыки:
\begin{itemize}
    \item создания нового проекта;
    \item добавления и настройки элементов управления;
    \item отладка корректного ввода данных для решения поставленной задачи;
    \item разработки алгоритма решения поставленной задачи с использованием
    оконного интерфейса;
    \item тестирования приложения;
    \item документирования разработанного кода.
\end{itemize}

%\section{Вычисление факториала}

\textbf{Задание:} Разработать приложение для вычисления факториала по приведенному примеру.

Создано окно приложения, содержащее два элемента TextBox, два элемента Label и один элемент Button. Для отображения сообщений об ошибках
в окно добавлен элемент ErrorProvider. Вид окна представлен на рисунке \ref{fig:task1_form}.
\begin{figure}[H]
    \centering
    \includegraphics{task1/form.png}
    \caption{Внешний вид формы программы для вычисления факториала}
    \label{fig:task1_form}
\end{figure}
У элементов изменены значения некоторых атрибутов. 
Значения измененных атрибутов представлены в таблице \ref{table:params1}.

\begin{table}[H]
    \small
    \caption{Значения атрибутов элементов в приложении <<Факториал>>}\label{tab:fact-attr}
    \begin{tabular}{|l|l|}\hline
    Наименование атрибута & Значение\cr\hline
    \multicolumn{2}{|l|}{Для формы}\cr\hline
    \verb"Text" & \verb"Форма"\cr\hline
    \verb"FormBorderStyle" & \verb"FixedSingle"\cr\hline
    \verb"MaximizeBox" & \verb"False"\cr\hline
    \multicolumn{2}{|l|}{Для первой надписи}\cr\hline
    \verb"(Name)" & \verb"lblInput"\cr\hline
    \verb"Text" & \verb"Введите целое число"\cr\hline
    \multicolumn{2}{|l|}{Для второй надписи}\cr\hline
    \verb"(Name)" & \verb"lblOutput"\cr\hline
    \verb"Text" & \verb"Результат"\cr\hline
    \multicolumn{2}{|l|}{Для первого текстового поля}\cr\hline
    \verb"(Name)" & \verb"txtInput"\cr\hline
    \multicolumn{2}{|l|}{Для второго текстового поля}\cr\hline
    \verb"(Name)" & \verb"txtOutput"\cr\hline
    \multicolumn{2}{|l|}{Для кнопки}\cr\hline
    \verb"(Name)" & \verb"btnCalculate"\cr\hline
    \verb"Text" & \verb"Вычислить"\cr\hline
    \multicolumn{2}{|l|}{Для обработчика ошибок}\cr\hline
    \verb"(Name)" & \verb"errorProvider1"\cr\hline
    \end{tabular}
    \label{table:params1}
\end{table}

Для работы программы была написана функция вычисления факториала:
\inputminted[fontsize=\small, breaklines=true, style=bw, linenos]{cpp}{task1/fact.h}
Здесь переменная \verb|N| "--- число, для которого нужно вычислить факториал.

На нажатие кнопки <<Вычислить>> установлено выполнение следующего
кода:
\begin{minted}[fontsize=\small, breaklines=true, style=bw, linenos]{cpp}
    private: System::Void btnCalculate_Click(System::Object^ sender, System::EventArgs^ e) {
		ClearAll();
		ll InputNumber;
		bool result = Int64::TryParse(txtInput->Text, InputNumber);
		if (!result) {
			errorProvider1->SetError(txtInput, "Введено не целое число");
			return;
		}
		if (InputNumber > 20) {
			errorProvider1->SetError(txtInput, "Число слишком большое");
			return;
		}
		ll OutputNumber = fact(InputNumber);
		if (OutputNumber == -1) {
			errorProvider1->SetError(txtInput, "Введено отрицательное число");
			return;
		}
		txtOutput->Text = System::Convert::ToString(OutputNumber);
	}
\end{minted}

После запуска приложения на экране появляется окно (см. рисунок \ref{fig:exec1})
\begin{figure}[H]
    \centering
    \includegraphics{task1/exec.png}
    \caption{Скриншот запуска программы}
    \label{fig:exec1}
\end{figure}

При вводе целого числа после нажатия кнопки в поле вывода приводится
результат вычисления факториала для заданного числа (см. рисунок \ref{fig:result1}).
\begin{figure}[H]
    \centering
    \includegraphics{task1/result.png}
    \caption{Результат работы}
    \label{fig:result1}
\end{figure}

Ввод некорректных значений обрабатывается элементом \verb|errorProvider1| и 
сопровождается сообщением об ошибке (см. рисунок \ref{fig:error1} )
\begin{figure}[H]
    \centering
    \includegraphics{task1/error.png}
    \caption{Сообщение об ошибке}
    \label{fig:error1}
\end{figure}
Полный код программы приведен в приложении А.
%\section{Простые вычисления}

\textbf{Задание:} Вычислить значение выражения ~$\displaystyle \frac{\cos{x} + \sin{y}}{\ln(x + y)}$
\vspace{0.25cm}

Создано окно, содержащее три элемента TextBox, три элемента Label и 
один элемент Button. Вид окна представлен на рисунке \ref{fig:form2}.

\begin{figure}[H]
    \centering
    \includegraphics{task2/form.png}
    \caption{Внешний вид формы в конструкторе}
    \label{fig:form2}
\end{figure}

У элементов изменены значения некоторых атрибутов. 
Значения измененных атрибутов представлены в таблице \ref{table:params2}.

\begin{table}[H]
    \small
    \caption{Значения атрибутов элементов в приложении <<Простые вычисления>>}\label{tab:fact-attr}
    \begin{tabular}{|l|l|}\hline
    Наименование атрибута & Значение\cr\hline
    \multicolumn{2}{|l|}{Для формы}\cr\hline
    \verb"Text" & \verb"Вычислить значение"\cr\hline
    \verb"FormBorderStyle" & \verb"FixedSingle"\cr\hline
    \verb"MaximizeBox" & \verb"False"\cr\hline
    \multicolumn{2}{|l|}{Для первой надписи}\cr\hline
    \verb"(Name)" & \verb"xLabel"\cr\hline
    \verb"Text" & \verb"Введите X"\cr\hline
    \multicolumn{2}{|l|}{Для второй надписи}\cr\hline
    \verb"(Name)" & \verb"yLabel"\cr\hline
    \verb"Text" & \verb"Введите Y"\cr\hline
    \multicolumn{2}{|l|}{Для третьей надписи}\cr\hline
    \verb"(Name)" & \verb"lblOutput"\cr\hline
    \verb"Text" & \verb"Результат:"\cr\hline
    \multicolumn{2}{|l|}{Для первого текстового поля}\cr\hline
    \verb"(Name)" & \verb"xInput"\cr\hline
    \multicolumn{2}{|l|}{Для второго текстового поля}\cr\hline
    \verb"(Name)" & \verb"yInput"\cr\hline
    \multicolumn{2}{|l|}{Для третьего текстового поля}\cr\hline
    \verb"(Name)" & \verb"txtOutput"\cr\hline
    \multicolumn{2}{|l|}{Для кнопки}\cr\hline
    \verb"(Name)" & \verb"btnCalculate"\cr\hline
    \verb"Text" & \verb"Вычислить"\cr\hline
    \multicolumn{2}{|l|}{Для обработчика ошибок}\cr\hline
    \verb"(Name)" & \verb"errorProvider1"\cr\hline
    \end{tabular}
    \label{table:params2}
\end{table}

Для работы программы была написана функция вычисления заданного выражения:
\inputminted[fontsize=\small, breaklines=true, style=bw, linenos]{cpp}{task2/f.h}
и функция, проверяющая выбранное поле на корректность:
\begin{minted}[fontsize=\small, breaklines=true, style=bw, linenos]{asm}
    bool VarValidation(System::Windows::Forms::TextBox^ Input, ll& x) { // Проверка числа для поля x из текста объекта Input
		
		ll InputNumber;
		bool IsVariableValid = Int64::TryParse(Input->Text, InputNumber); // Пробуем записать в InputNumber число из потока
		
		if (!IsVariableValid) { // Если не вышло
			errorProvider1->SetError(Input, "Переменная не целое число");
			return 0;
		}
		x = InputNumber; // Обновляем значение переменной
		return 1;
    }
\end{minted}
Логику работы программы реализует фрагмент кода, привязанный к кнопке
\begin{minted}[fontsize=\small, breaklines=true, style=bw, linenos]{cpp}
	private: System::Void btnCalculate_Click(System::Object^ sender, System::EventArgs^ e) {
		ClearAll();

		ll x = 0;
		ll y = 0;

		bool XisOkay = VarValidation(xInput, x); // Проверяем корректность x
		bool YisOkay = VarValidation(yInput, y); // Проверяем корректность y

		if (!XisOkay || !YisOkay) return; // Если какая-то из них некорректна - завершим работу. Все необходимые выводы исключений уже были произведены

		ll summary = x + y; // Считаем сумму для логарифма

		if (summary == 1) { // Если аргумент равен единице - логарифм равен нулю
			errorProvider1->SetError(txtOutput, "Деление на ноль");
			return;
		}

		if (summary <= 0) { // Если аргумент меньше нуля - неверно
			errorProvider1->SetError(txtOutput, "Недопустимое значение для логарифма");
			return;
		}

		double OutputNumber = f(x, y); // Получаем значение функции для заданных чисел х, у

		txtOutput->Text = System::Convert::ToString(OutputNumber); // Отображаем ответ
	}
\end{minted}

Функция \verb|ClearAll()| реализует очищение полей от ошибок.
После запуска приложения на экране появляется окно (см. рисунок \ref{fig:exec2})
\begin{figure}[H]
    \centering
    \includegraphics{task2/exec.png}
    \caption{Скриншот запуска программы}
    \label{fig:exec2}
\end{figure}
При вводе целого числа после нажатия кнопки в поле вывода приводится
результат вычисления факториала для заданного числа (см. рисунок \ref{fig:result2}).
\begin{figure}[H]
    \centering
    \includegraphics{task2/result.png}
    \caption{Результат работы}
    \label{fig:result2}
\end{figure}
Ввод некорректных значений обрабатывается элементом \verb|errorProvider1| и 
сопровождается сообщением об ошибке (см. рисунок \ref{fig:error2} )
\begin{figure}[H]
    \centering
    \includegraphics{task2/error.png}
    \caption{Сообщение об ошибке}
    \label{fig:error2}
\end{figure}
Полный код программы приведен в приложении А.
%\section{Рекурсивные вычисления}
\textbf{Задание:} Создать рекурсивную функцию, которая для заданного целого $n$
вычисляет сумму ряда $\displaystyle \sum\limits_{i=1}^n 2^i$

Для этого были использовано 3 элемента Label, 2 элемента TextBox, 1 элемент PictureBox
и один элемент Button. 

Создано окно, содержащее три элемента TextBox, три элемента Label и 
один элемент Button. После запуска приложения появляется окно (см. рисунок \ref{fig:form3}) .
\begin{figure}[H]
    \centering
    \includegraphics{task3/form.png}
    \caption{Внешний вид формы в конструкторе}
    \label{fig:form3}
\end{figure}

У элементов изменены значения некоторых атрибутов. 
Значения измененных атрибутов представлены в таблице \ref{table:params3}.

\begin{table}[H]
    \small
    \caption{Значения атрибутов элементов в приложении <<Рекурсивные вычисления>>}
    \begin{tabular}{|l|l|}\hline
    Наименование атрибута & Значение\cr\hline
    \multicolumn{2}{|l|}{Для формы}\cr\hline
    \verb"Text" & \verb"Вычислить значение"\cr\hline
    \verb"FormBorderStyle" & \verb"FixedSingle"\cr\hline
    \verb"MaximizeBox" & \verb"False"\cr\hline
    \multicolumn{2}{|l|}{Для первой надписи}\cr\hline
    \verb"(Name)" & \verb"formulaText"\cr\hline
    \verb"Text" & \verb"Вычисление выражения"\cr\hline
    \multicolumn{2}{|l|}{Для второй надписи}\cr\hline
    \verb"(Name)" & \verb"xLabel"\cr\hline
    \verb"Text" & \verb"Введите N"\cr\hline
    \multicolumn{2}{|l|}{Для третьей надписи}\cr\hline
    \verb"(Name)" & \verb"lblOutput"\cr\hline
    \verb"Text" & \verb"Результат:"\cr\hline
    \multicolumn{2}{|l|}{Для первого текстового поля}\cr\hline
    \verb"(Name)" & \verb"xInput"\cr\hline
    \multicolumn{2}{|l|}{Для второго текстового поля}\cr\hline
    \verb"(Name)" & \verb"txtOutput"\cr\hline
    \multicolumn{2}{|l|}{Для кнопки}\cr\hline
    \verb"(Name)" & \verb"btnCalculate"\cr\hline
    \verb"Text" & \verb"Вычислить"\cr\hline
    \multicolumn{2}{|l|}{Для обработчика ошибок}\cr\hline
    \verb"(Name)" & \verb"errorProvider1"\cr\hline
    \multicolumn{2}{|l|}{Для изображения выражения}\cr\hline
    \verb"(Name)" & \verb"pictureBox1"\cr\hline
    \end{tabular}
    \label{table:params3}
\end{table}

Программа содержит функции (\verb|ClearAll()|, \verb|VarValidation(System::Windows::Forms::TextBox^ Input, ll& x))|, 
аналогичные функциям из Задания 2 за исключением логики работы кнопки

\begin{minted}[fontsize=\small, breaklines=true, style=bw, linenos]{cpp}
    private: System::Void btnCalculate_Click(System::Object^ sender, System::EventArgs^ e) {
		ClearAll();

		ll n = 0;

		bool NisOkay = VarValidation(nInput, n); // Проверяем корректность n

		if (n <= 0) {
			errorProvider1->SetError(nInput, "Недопустимая степень");
			return;
		}
			

		if (!NisOkay) return; // Если число некорректно - завершим работу. Все необходимые выводы исключений уже были произведены

		ll OutputNumber = f(n); // Получаем значение ряда для заданного n

		txtOutput->Text = System::Convert::ToString(OutputNumber); // Отображаем ответ
	}
\end{minted}
и функции подсчета суммы ряда
\inputminted[fontsize=\small, breaklines=true, style=bw, linenos]{cpp}{task3/f.h}
%\section{Обработка табличных данных. Часть 1.}

\textbf{Задание:} Найти сумму нечетных элементов, меньших заданного числа. 
Вывести максимальный четный элемент. 

Создано окно приложения, содержащее два элемента TextBox, два элемента Label, 4 элемента Button, и 1 элемент DataGridView. 
Для отображения сообщений об ошибках в окно добавлен элемент ErrorProvider. Вид окна представлен на рисунке \ref{fig:task4_form}.

\begin{figure}[H]
    \centering
    \includegraphics{task4/form.png}
    \caption{Внешний вид формы программы}
    \label{fig:task4_form}
\end{figure}

У элементов изменены значения некоторых атрибутов. 
Значения измененных атрибутов представлены в таблице \ref{table:params2}.

\begin{table}[H]
    \small
    \caption{Значения атрибутов элементов в приложении <<Обработка табличных данных. Часть 1.>>}\label{tab:fact-attr}
    \begin{tabular}{|l|l|}\hline
    Наименование атрибута & Значение\cr\hline
    \multicolumn{2}{|l|}{Для формы}\cr\hline
    \verb"Text" & \verb"Браузер "Амиго""\cr\hline
    \verb"FormBorderStyle" & \verb"FixedSingle"\cr\hline
    \verb"MaximizeBox" & \verb"False"\cr\hline
    \multicolumn{2}{|l|}{Для первой надписи}\cr\hline
    \verb"(Name)" & \verb"xLabel"\cr\hline
    \verb"Text" & \verb"Введите X"\cr\hline
    \multicolumn{2}{|l|}{Для второй надписи}\cr\hline
    \verb"(Name)" & \verb"lblOutput"\cr\hline
    \verb"Text" & \verb"Вывод"\cr\hline
    \multicolumn{2}{|l|}{Для первого текстового поля}\cr\hline
    \verb"(Name)" & \verb"txtX"\cr\hline
    \multicolumn{2}{|l|}{Для второго текстового поля}\cr\hline
    \verb"(Name)" & \verb"txtOutput"\cr\hline
    \multicolumn{2}{|l|}{Для кнопки суммирования}\cr\hline
    \verb"(Name)" & \verb"btnSummary"\cr\hline
    \verb"Text" & \verb"Сумма на интервале, удовлетворяющие заданному условию"\cr\hline
    \multicolumn{2}{|l|}{Для кнопки для нахождения максимального четного элемента}\cr\hline
    \verb"(Name)" & \verb"btnFindMaxEven"\cr\hline
    \verb"Text" & \verb"Максимальный четный элемент"\cr\hline
    \multicolumn{2}{|l|}{Для кнопки для нахождения максимального четного элемента}\cr\hline
    \verb"(Name)" & \verb"btnFindMaxEven"\cr\hline
    \verb"Text" & \verb"Максимальный четный элемент"\cr\hline
    \multicolumn{2}{|l|}{Для кнопки добавления ряда}\cr\hline
    \verb"(Name)" & \verb"btnAdd"\cr\hline
    \verb"Text" & \verb"Добавить"\cr\hline
    \multicolumn{2}{|l|}{Для кнопки удаления ряда}\cr\hline
    \verb"(Name)" & \verb"btnRemove"\cr\hline
    \verb"Text" & \verb"Удалить"\cr\hline
    \multicolumn{2}{|l|}{Для таблицы}\cr\hline
    \verb"(Name)" & \verb"grid"\cr\hline
    \verb"(Name)" & \verb"grid"\cr\hline
    \multicolumn{2}{|l|}{Для обработчика ошибок}\cr\hline
    \verb"(Name)" & \verb"errorProvider1"\cr\hline
    \end{tabular}
    \label{table:params2}
\end{table}

Для изменения количества рядов в таблице были реализованы кнопки добавления
и удаления ряда. Код соответствующих функций приведен ниже:

\begin{minted}[fontsize=\small, breaklines=true, style=bw, linenos]{cpp}
    private: System::Void btnAdd_Click(System::Object^ sender, System::EventArgs^ e) { // Добавление
			this->grid->Rows->Add(1);
	}
    private: System::Void btnRemove_Click(System::Object^ sender, System::EventArgs^ e) { // Удаление
		if (!this->grid->CurrentRow->IsNewRow) {
			int i = this->grid->CurrentRow->Index;
			this->grid->Rows->Remove(this->grid->Rows[i]);
		}
	}

\end{minted}

Решение каждой из двух задач реализовано в соответствующих кнопках. Код 
представлен ниже:

\begin{minted}[fontsize=\small, breaklines=true, style=bw, linenos]{cpp}
        private: System::Void btnSummary_Click(System::Object^ sender, System::EventArgs^ e) { // Вычисление необходимой суммы
			ClearOutput();
			bool noBadCells = true;
			bool check2 = true;
			if ((int)WrongCells->size() > 0) {
				noBadCells = false;
			}
			int xValue;
			bool isCorrectValue = Int32::TryParse(txtX->Text, xValue);
			if (!isCorrectValue) {
				errorProvider1->SetError(txtX, "Неверное значение для X");
			}
			if (!isCorrectValue) {
				return;
			}
			else {
				errorProvider1->SetError(txtX, String::Empty);
			}
			if (!noBadCells) {
				return;
			}
			int summary = 0;
			for (int i = 0; i < this->grid->RowCount; ++i) {
				int val = System::Convert::ToInt32(this->grid->Rows[i]->Cells[0]->Value);
				if (val % 2 == 1 && val < xValue) {
					summary += val;
				}

			}
			ClearOutput();
			txtOutput->Text = System::Convert::ToString(summary);
		}
		private: System::Void btnFindMaxEven_Click(System::Object^ sender, System::EventArgs^ e) { // Вывод необходимого максимального элемента на экран
			ClearOutput();
			if (WrongCells->size() > 0) {
				return;
			}
			int max_val = -1e9;
			int max_ind = -1;
			for (int i = 0; i < this->grid->RowCount; ++i) {
				int val = System::Convert::ToInt32(this->grid->Rows[i]->Cells[0]->Value);
				if (val > max_val && val % 2 == 0) {
					max_val = val;
					max_ind = i;
				}
			}
			ClearOutput();
			txtOutput->Text = System::Convert::ToString(max_val);
		}
\end{minted}

Контроль за корректностью введенных данных осуществляется через поддерживание
невалидных ячеек в set из библиотеки STL. Работа с ней осуществляется через 
обработку события CellLeave:

\begin{minted}[fontsize=\small, breaklines=true, style=bw, linenos]{cpp}
    private: System::Void grid_CellLeave(System::Object^ sender, System::Windows::Forms::DataGridViewCellEventArgs^ e) {
			int val = 0;

			System::String^ Value = System::Convert::ToString(this->grid->Rows[grid->CurrentRow->Index]->Cells[0]->EditedFormattedValue);
			
			bool isCorrectValue = Int32::TryParse(Value, val); // Пробуем записать в InputNumber число из потока
			if (!isCorrectValue && (Value != "" && grid->CurrentRow->Index != grid->RowCount)) {
				errorProvider1->SetError(grid, "Неверный тип для ячейки");
				WrongCells->insert(grid->CurrentRow->Index);
			}
			else {
				WrongCells->erase(grid->CurrentRow->Index);
				if ((int)WrongCells->size() == 0) {
					errorProvider1->SetError(grid, String::Empty);
				}
			}
			
		}
\end{minted}

После запуска приложения на экране появляется окно (см. рисунок \ref{fig:exec4})
\begin{figure}[H]
    \centering
    \includegraphics{task4/exec.png}
    \caption{Скриншот запуска программы}
    \label{fig:exec4}
\end{figure}

При вводе целого числа после нажатия кнопки в поле вывода приводится
результат вычисления суммы в заданном интервале (см. рисунок \ref{fig:result41}).
\begin{figure}[H]
    \centering
    \includegraphics{task4/result1.png}
    \caption{Результат работы}
    \label{fig:result41}
\end{figure}

При вводе целого числа после нажатия кнопки в поле вывода приводится
результат вычисления максимального четного элемента (см. рисунок \ref{fig:result42}).
\begin{figure}[H]
    \centering
    \includegraphics{task4/result2.png}
    \caption{Результат работы}
    \label{fig:result42}
\end{figure}

Ввод некорректных значений обрабатывается элементом \verb|errorProvider1| и 
сопровождается сообщением об ошибке (см. рисунок \ref{fig:error4} )
\begin{figure}[H]
    \centering
    \includegraphics{task4/error.png}
    \caption{Сообщение об ошибке}
    \label{fig:error4}
\end{figure}
Полный код программы приведен в приложении А.
%\section{Обработка табличных данных. Часть 2.}

\textbf{Задание:} Ваш выбор: Все нечетные столбцы заменить столбцом X. 
(Нумерация столбцов массива начинается с нуля.)

Создано окно приложения, содержащее 5 элементОв Button, 3 элемента DataGridView и 5 элементов Label. 
Для отображения сообщений об ошибках в окно добавлен элемент ErrorProvider. Вид окна представлен на рисунке \ref{fig:task5_form}.

\begin{figure}[H]
    \centering
    \includegraphics{task5/form.png}
    \caption{Внешний вид формы программы}
    \label{fig:task5_form}
\end{figure}

У элементов изменены значения некоторых атрибутов. 
Значения измененных атрибутов представлены в таблице \ref{table:params5}.

\begin{table}[H]
    \small
    \caption{Значения атрибутов элементов в приложении <<Обработка табличных данных. Часть 2.>>}
    \begin{tabular}{|l|l|}\hline
    Наименование атрибута & Значение\cr\hline
    \multicolumn{2}{|l|}{Для формы}\cr\hline
    \verb"Text" & \verb"Браузер "Амиго""\cr\hline
    \verb"FormBorderStyle" & \verb"FixedSingle"\cr\hline
    \verb"MaximizeBox" & \verb"False"\cr\hline
    \multicolumn{2}{|l|}{Для первой надписи}\cr\hline
    \verb"(Name)" & \verb"taskLabel"\cr\hline
    \verb"Text" & \verb"Задание: Все нечетные столбцы заменить столбцом X. (Нумерация столбцов массива начинается с нуля."\cr\hline
    \multicolumn{2}{|l|}{Для второй надписи}\cr\hline
    \verb"(Name)" & \verb"initLabel"\cr\hline
    \verb"Text" & \verb"Исходная матрица"\cr\hline
    \multicolumn{2}{|l|}{Для третьей надписи}\cr\hline
    \verb"(Name)" & \verb"xLabel"\cr\hline
    \verb"Text" & \verb"Столбец X"\cr\hline
    \multicolumn{2}{|l|}{Для четвертой надписи}\cr\hline
    \verb"(Name)" & \verb"resultLabel"\cr\hline
    \verb"Text" & \verb"Результат"\cr\hline
    \multicolumn{2}{|l|}{Для кнопки "Выполнить"}\cr\hline
    \verb"(Name)" & \verb"btnCalc"\cr\hline
    \multicolumn{2}{|l|}{Для кнопки добавления ряда}\cr\hline
    \verb"(Name)" & \verb"btnAddRow"\cr\hline
    \verb"Text" & \verb"Добавить"\cr\hline
    \multicolumn{2}{|l|}{Для кнопки удаления ряда}\cr\hline
    \verb"(Name)" & \verb"btnRemoveRow"\cr\hline
    \verb"Text" & \verb"Удалить"\cr\hline
    \multicolumn{2}{|l|}{Для кнопки добавления столбца}\cr\hline
    \verb"(Name)" & \verb"btnAddColumn"\cr\hline
    \verb"Text" & \verb"Добавить"\cr\hline
    \multicolumn{2}{|l|}{Для кнопки удаления столбца}\cr\hline
    \verb"(Name)" & \verb"btnRemoveColumn"\cr\hline
    \verb"Text" & \verb"Удалить"\cr\hline
    \multicolumn{2}{|l|}{Для обработчика ошибок}\cr\hline
    \verb"(Name)" & \verb"errorProvider1"\cr\hline
    \end{tabular}
    \label{table:params5}
\end{table}

Для изменения количества рядов в таблице были реализованы кнопки добавления
и удаления ряда. Код соответствующих функций приведен ниже:

\begin{minted}[fontsize=\small, breaklines=true, style=bw, linenos]{cpp}
    private: System::Void btnAdd_Click(System::Object^ sender, System::EventArgs^ e) { // Добавление
			this->grid->Rows->Add(1);
	}
    private: System::Void btnRemove_Click(System::Object^ sender, System::EventArgs^ e) { // Удаление
		if (!this->grid->CurrentRow->IsNewRow) {
			int i = this->grid->CurrentRow->Index;
			this->grid->Rows->Remove(this->grid->Rows[i]);
		}
	}

\end{minted}

Решение каждой из двух задач реализовано в соответствующих кнопках. Код 
представлен ниже:

\begin{minted}[fontsize=\small, breaklines=true, style=bw, linenos]{cpp}
        private: System::Void btnSummary_Click(System::Object^ sender, System::EventArgs^ e) { // Вычисление необходимой суммы
			ClearOutput();
			bool noBadCells = true;
			bool check2 = true;
			if ((int)WrongCells->size() > 0) {
				noBadCells = false;
			}
			int xValue;
			bool isCorrectValue = Int32::TryParse(txtX->Text, xValue);
			if (!isCorrectValue) {
				errorProvider1->SetError(txtX, "Неверное значение для X");
			}
			if (!isCorrectValue) {
				return;
			}
			else {
				errorProvider1->SetError(txtX, String::Empty);
			}
			if (!noBadCells) {
				return;
			}
			int summary = 0;
			for (int i = 0; i < this->grid->RowCount; ++i) {
				int val = System::Convert::ToInt32(this->grid->Rows[i]->Cells[0]->Value);
				if (val % 2 == 1 && val < xValue) {
					summary += val;
				}

			}
			ClearOutput();
			txtOutput->Text = System::Convert::ToString(summary);
		}
		private: System::Void btnFindMaxEven_Click(System::Object^ sender, System::EventArgs^ e) { // Вывод необходимого максимального элемента на экран
			ClearOutput();
			if (WrongCells->size() > 0) {
				return;
			}
			int max_val = -1e9;
			int max_ind = -1;
			for (int i = 0; i < this->grid->RowCount; ++i) {
				int val = System::Convert::ToInt32(this->grid->Rows[i]->Cells[0]->Value);
				if (val > max_val && val % 2 == 0) {
					max_val = val;
					max_ind = i;
				}
			}
			ClearOutput();
			txtOutput->Text = System::Convert::ToString(max_val);
		}
\end{minted}

Контроль за корректностью введенных данных осуществляется через поддерживание
невалидных ячеек в set из библиотеки STL. Работа с ней осуществляется через 
обработку события CellLeave:

\begin{minted}[fontsize=\small, breaklines=true, style=bw, linenos]{cpp}
    private: System::Void grid_CellLeave(System::Object^ sender, System::Windows::Forms::DataGridViewCellEventArgs^ e) {
			int val = 0;

			System::String^ Value = System::Convert::ToString(this->grid->Rows[grid->CurrentRow->Index]->Cells[0]->EditedFormattedValue);
			
			bool isCorrectValue = Int32::TryParse(Value, val); // Пробуем записать в InputNumber число из потока
			if (!isCorrectValue && (Value != "" && grid->CurrentRow->Index != grid->RowCount)) {
				errorProvider1->SetError(grid, "Неверный тип для ячейки");
				WrongCells->insert(grid->CurrentRow->Index);
			}
			else {
				WrongCells->erase(grid->CurrentRow->Index);
				if ((int)WrongCells->size() == 0) {
					errorProvider1->SetError(grid, String::Empty);
				}
			}
			
		}
\end{minted}

После запуска приложения на экране появляется окно (см. рисунок \ref{fig:exec4})
\begin{figure}[H]
    \centering
    \includegraphics{task4/exec.png}
    \caption{Скриншот запуска программы}
    \label{fig:exec4}
\end{figure}

При вводе целого числа после нажатия кнопки в поле вывода приводится
результат вычисления суммы в заданном интервале (см. рисунок \ref{fig:result41}).
\begin{figure}[H]
    \centering
    \includegraphics{task4/result1.png}
    \caption{Результат работы}
    \label{fig:result41}
\end{figure}

При вводе целого числа после нажатия кнопки в поле вывода приводится
результат вычисления максимального четного элемента (см. рисунок \ref{fig:result42}).
\begin{figure}[H]
    \centering
    \includegraphics{task4/result2.png}
    \caption{Результат работы}
    \label{fig:result42}
\end{figure}

Ввод некорректных значений обрабатывается элементом \verb|errorProvider1| и 
сопровождается сообщением об ошибке (см. рисунок \ref{fig:error4} )
\begin{figure}[H]
    \centering
    \includegraphics{task4/error.png}
    \caption{Сообщение об ошибке}
    \label{fig:error4}
\end{figure}
Полный код программы приведен в приложении А.
%\section{Матричный калькулятор}

\textbf{Задание:} Создать приложение, реализующее основные операции с векторами и матрицами:

\begin{enumerate}
    \item Ввод матрицы, вектора
    \item Создание матриц (единичная, матрица как набор векторов)
    \item Умножение на число, вектор, матрицу
    \item Сложение/вычитание двух матриц
    \item Сложение/вычитание двух векторов
    \item Скалярное и векторное произведение двух векторов
    \item Транспонированная матрица
    \item Определитель, ранг матрицы
\end{enumerate}
Выводить сообщения об ошибках (ввод не числа, несоответствие размерностей)

Интерфейс приложения можно условно разделить на четыре части, две из которых
отвечают за определение соответствующих аргументов операции, третья
за непосредственно саму операцию, а четвертая графически отображает
аргументы и результат

Вид окна представлен на рисунке \ref{fig:task6_form}.
\begin{figure}[H]
    \centering
    \includegraphics[scale=0.7]{task6/form.png}
    \caption{Внешний вид формы программы}
    \label{fig:task6_form}
\end{figure}

У элементов изменены значения некоторых атрибутов. 
Значения измененных атрибутов представлены в таблице \ref{table:params6}.

\begin{longtable}{|l|l|}
    Наименование атрибута & Значение\cr\hline
    \multicolumn{2}{|l|}{Для формы}\cr\hline
    \verb"Text" & \verb"Браузер "Амиго""\cr\hline
    \verb"FormBorderStyle" & \verb"FixedSingle"\cr\hline
    \verb"MaximizeBox" & \verb"False"\cr\hline
    \multicolumn{2}{|l|}{Для первой группы}\cr\hline
    \verb"(Name)" & \verb"MatrixGroup1"\cr\hline
    \multicolumn{2}{|l|}{Для второй группы}\cr\hline
    \verb"(Name)" & \verb"MatrixGroup2"\cr\hline
    \multicolumn{2}{|l|}{Для третьей группы}\cr\hline
    \verb"(Name)" & \verb"Result_Group"\cr\hline
    \multicolumn{2}{|l|}{Для надписи первой матрицы}\cr\hline
    \verb"(Name)" & \verb"initLabel"\cr\hline
    \verb"Text" & \verb"Матрица 1"\cr\hline
    \multicolumn{2}{|l|}{Для надписи второй матрицы}\cr\hline
    \verb"(Name)" & \verb"xLabel"\cr\hline
    \verb"Text" & \verb"Матрица 2"\cr\hline
    \multicolumn{2}{|l|}{Для третьей надписи}\cr\hline
    \verb"(Name)" & \verb"resultLabel"\cr\hline
    \verb"Text" & \verb"Результат"\cr\hline
    \multicolumn{2}{|l|}{Для радиокнопки "Матрица" для группы 1}\cr\hline
    \verb"(Name)" & \verb"Grid1_MatrixRBtn"\cr\hline
    \verb"Text" & \verb"Матрица"\cr\hline
    \multicolumn{2}{|l|}{Для радиокнопки "Вектор" для группы 1}\cr\hline
    \verb"(Name)" & \verb"Grid1_VectorRBtn"\cr\hline
    \verb"Text" & \verb"Вектор"\cr\hline
    \multicolumn{2}{|l|}{Для радиокнопки "Число" для группы 1}\cr\hline
    \verb"(Name)" & \verb"Grid1_NumRBtn"\cr\hline
    \verb"Text" & \verb"Число"\cr\hline
    \multicolumn{2}{|l|}{Для радиокнопки "Матрица" для группы 2}\cr\hline
    \verb"(Name)" & \verb"Grid2_MatrixRBtn"\cr\hline
    \verb"Text" & \verb"Матрица"\cr\hline
    \multicolumn{2}{|l|}{Для радиокнопки "Вектор" для группы 2}\cr\hline
    \verb"(Name)" & \verb"Grid2_VectorRBtn"\cr\hline
    \verb"Text" & \verb"Вектор"\cr\hline
    \multicolumn{2}{|l|}{Для радиокнопки "Число" для группы 2}\cr\hline
    \verb"(Name)" & \verb"Grid2_NumRBtn"\cr\hline
    \verb"Text" & \verb"Число"\cr\hline
    \multicolumn{2}{|l|}{Для радиокнопки транспонирования для группы 3}\cr\hline
    \verb"(Name)" & \verb"Transposition_RBtn"\cr\hline
    \verb"Text" & \verb"Транспон. матрицы (для кв. матриц)"\cr\hline
    \multicolumn{2}{|l|}{Для радиокнопки ранга матрицы для группы 3}\cr\hline
    \verb"(Name)" & \verb"Rank_RBtn"\cr\hline
    \verb"Text" & \verb"Ранг матрицы"\cr\hline
    \multicolumn{2}{|l|}{Для радиокнопки суммы для группы 3}\cr\hline
    \verb"(Name)" & \verb"Sum_RBtn"\cr\hline
    \verb"Text" & \verb"Сумма"\cr\hline
    \multicolumn{2}{|l|}{Для радиокнопки разности для группы 3}\cr\hline
    \verb"(Name)" & \verb"Difference_RBtn"\cr\hline
    \verb"Text" & \verb"Разность"\cr\hline
    \multicolumn{2}{|l|}{Для радиокнопки произведения для группы 3}\cr\hline
    \verb"(Name)" & \verb"Mult_RBtn"\cr\hline
    \verb"Text" & \verb"Произведение"\cr\hline
    \multicolumn{2}{|l|}{Для радиокнопки определителя для группы 3}\cr\hline
    \verb"(Name)" & \verb"Det_RBtn"\cr\hline
    \verb"Text" & \verb"Определитель (для кв. матриц) "\cr\hline
    \multicolumn{2}{|l|}{Для радиокнопки скалярного произведения для группы 3}\cr\hline
    \verb"(Name)" & \verb"ScalarMultiply_RBtn"\cr\hline
    \verb"Text" & \verb"Скалярное произведение (для векторов) "\cr\hline
    \multicolumn{2}{|l|}{Для радиокнопки векторного произведения для группы 3}\cr\hline
    \verb"(Name)" & \verb"VectorMultiply_RBtn"\cr\hline
    \verb"Text" & \verb"Векторное произведение (для трехмерных векторов)"\cr\hline
    \multicolumn{2}{|l|}{Для кнопки создания единичной матрицы для группы 1}\cr\hline
    \verb"(Name)" & \verb"CreateMatrix1_Btn"\cr\hline
    \verb"Text" & \verb"Создать единичную матрицу размера"\cr\hline
    \multicolumn{2}{|l|}{Для кнопки создания единичной матрицы для группы 2}\cr\hline
    \verb"(Name)" & \verb"CreateMatrix2_Btn"\cr\hline
    \verb"Text" & \verb"Создать единичную матрицу размера"\cr\hline
    \multicolumn{2}{|l|}{Для знака размерности для группы 1}\cr\hline
    \verb"(Name)" & \verb"Scale_label1"\cr\hline
    \verb"Text" & \verb"X"\cr\hline
    \multicolumn{2}{|l|}{Для знака размерности для группы 2}\cr\hline
    \verb"(Name)" & \verb"Scale_label2"\cr\hline
    \verb"Text" & \verb"X"\cr\hline
    \multicolumn{2}{|l|}{Для текстового поля N размерности для группы 1}\cr\hline
    \verb"(Name)" & \verb"N1_TextBox"\cr\hline
    \multicolumn{2}{|l|}{Для текстового поля M размерности для группы 1}\cr\hline
    \verb"(Name)" & \verb"M1_TextBox"\cr\hline
    \multicolumn{2}{|l|}{Для текстового поля N размерности для группы 2}\cr\hline
    \verb"(Name)" & \verb"N2_TextBox"\cr\hline
    \multicolumn{2}{|l|}{Для текстового поля M размерности для группы 2}\cr\hline
    \verb"(Name)" & \verb"M2_TextBox"\cr\hline

    \multicolumn{2}{|l|}{Для кнопки "Добавить строку" для группы 1}\cr\hline
    \verb"(Name)" & \verb"Grid1_btnAddRow"\cr\hline
    \verb"Text" & \verb"Добавить строку"\cr\hline
    \multicolumn{2}{|l|}{Для кнопки "Добавить столбец" для группы 1}\cr\hline
    \verb"(Name)" & \verb"Grid1_btnAddColumn"\cr\hline
    \verb"Text" & \verb"Добавить столбец"\cr\hline
    \multicolumn{2}{|l|}{Для кнопки "Удалить строку" для группы 1}\cr\hline
    \verb"(Name)" & \verb"Grid1_btnRmvRow"\cr\hline
    \verb"Text" & \verb"Удалить строку"\cr\hline
    \multicolumn{2}{|l|}{Для кнопки "Удалить столбец" для группы 1}\cr\hline
    \verb"(Name)" & \verb"Grid1_btnRmvColumn"\cr\hline
    \verb"Text" & \verb"Удалить столбец"\cr\hline
    \multicolumn{2}{|l|}{Для кнопки "Добавить строку" для группы 2}\cr\hline
    \verb"(Name)" & \verb"Grid2_btnAddRow"\cr\hline
    \verb"Text" & \verb"Добавить строку"\cr\hline
    \multicolumn{2}{|l|}{Для кнопки "Добавить столбец" для группы 2}\cr\hline
    \verb"(Name)" & \verb"Grid2_btnAddColumn"\cr\hline
    \verb"Text" & \verb"Добавить столбец"\cr\hline
    \multicolumn{2}{|l|}{Для кнопки "Удалить строку" для группы 2}\cr\hline
    \verb"(Name)" & \verb"Grid2_btnRmvRow"\cr\hline
    \verb"Text" & \verb"Удалить строку"\cr\hline
    \multicolumn{2}{|l|}{Для кнопки "Удалить столбец" для группы 2}\cr\hline
    \verb"(Name)" & \verb"Grid2_btnRmvColumn"\cr\hline
    \verb"Text" & \verb"Удалить столбец"\cr\hline
    \multicolumn{2}{|l|}{Для кнопки добавления ряда}\cr\hline
    \verb"(Name)" & \verb"btnAddRow"\cr\hline
    \verb"Text" & \verb"Добавить"\cr\hline
    \multicolumn{2}{|l|}{Для кнопки удаления ряда}\cr\hline
    \verb"(Name)" & \verb"btnRemoveRow"\cr\hline
    \verb"Text" & \verb"Удалить"\cr\hline
    \multicolumn{2}{|l|}{Для кнопки добавления столбца}\cr\hline
    \verb"(Name)" & \verb"btnAddColumn"\cr\hline
    \verb"Text" & \verb"Добавить"\cr\hline
    \multicolumn{2}{|l|}{Для кнопки удаления столбца}\cr\hline
    \verb"(Name)" & \verb"btnRemoveColumn"\cr\hline
    \verb"Text" & \verb"Удалить"\cr\hline
    \multicolumn{2}{|l|}{Для обработчика ошибок}\cr\hline
    \verb"(Name)" & \verb"errorProvider1"\cr\hline

    \caption{Значения атрибутов элементов в приложении <<Матричный калькулятор}
    \label{table:params6}
\end{longtable}

Программа работает за счет дополнительно написанной библиотеки для работы с матрицами 
с помощью \verb|std::vector| библиотеки STL. В ней реализованы все операции.

Полный код библиотеки приведен в приложении B.

Кроме того, написаны две вспомогательный функции, представляющие из себя связки
между \verb|std::vector| и \verb|DataGridView|. Код приведен ниже:

\inputminted[fontsize=\small, breaklines=true, style=bw, linenos, encoding=cp1251, outencoding=utf8]{cpp}{task6/Misc.h}

Примеры работы приведены ниже:

\begin{figure}[H]
    \centering
    \includegraphics[scale=0.4]{task6/det.png}
    \caption{Пример нахождения определителя матрицы}
\end{figure}

\begin{figure}[H]
    \centering
    \includegraphics[scale=0.4]{task6/mult.png}   
    \caption{Пример нахождения векторного произведения}
\end{figure}

\begin{figure}[H]
    \centering
    \includegraphics[scale=0.4]{task6/scalar.png}
    \caption{Пример нахождения скалярного произведения векторов}
\end{figure}

Можно заметить, что интерфейс приложения динамически изменяется в 
зависимости от выбранных параметров в соответствующих радиокнопках.

Контроль за корректностью введенных данных осуществляется внутри соответствующих
функций. 

В случае, если вводятся некорректные данные или операция не поддерживается
над операндами этого типа "--- будет выведено сообщение об ошибке:
\begin{figure}[H]
    \centering
    \includegraphics[scale=0.4]{task6/error1.png}
    \caption{Пример обработки ошибки}
\end{figure}
Полный код программы приведен в приложении А.


%\section{Использование коллекций}

\textbf{Задание:} Создать словарь, состоящий из строк. В качестве ключа выступает фамилия, в качестве значения — должность. Вывести на экран фамилии людей, занимающих данную должность. 
Вывести должность, занимаемую данным человеком.

Вид окна представлен на рисунке \ref{fig:task7_form}.
\begin{figure}[H]
    \centering
    \includegraphics[scale=0.7]{task7/form.png}
    \caption{Внешний вид формы программы}
    \label{fig:task7_form}
\end{figure}

У элементов изменены значения некоторых атрибутов. 
Значения измененных атрибутов представлены в таблице \ref{table:params7}.

\begin{longtable}{|l|l|}
    Наименование атрибута & Значение\cr\hline
    \multicolumn{2}{|l|}{Для формы}\cr\hline
    \verb"Text" & \verb"Браузер "Амиго""\cr\hline
    \verb"FormBorderStyle" & \verb"FixedSingle"\cr\hline
    \verb"MaximizeBox" & \verb"False"\cr\hline
    \multicolumn{2}{|l|}{Для надписи ФИО}\cr\hline
    \verb"(Name)" & \verb"NameLabel"\cr\hline
    \verb"Text" & \verb"ФИО (ключ)"\cr\hline
    \multicolumn{2}{|l|}{Для текстового поля должности}\cr\hline
    \verb"(Name)" & \verb"PositionTextBox"\cr\hline
    \multicolumn{2}{|l|}{Для текстового поля ФИО}\cr\hline
    \verb"(Name)" & \verb"NameTextBox"\cr\hline
    \multicolumn{2}{|l|}{Для надписи должности}\cr\hline
    \verb"(Name)" & \verb"PositionLabel"\cr\hline
    \verb"Text" & \verb"Должность(значение)"\cr\hline
    \multicolumn{2}{|l|}{Для надписи к таблице}\cr\hline
    \verb"(Name)" & \verb"resultLabel"\cr\hline
    \verb"Text" & \verb"Список сотрудников"\cr\hline
    \multicolumn{2}{|l|}{Для кнопки добавления значения по ключу}\cr\hline
    \verb"(Name)" & \verb"AddBtn"\cr\hline
    \verb"Text" & \verb"Добавить"\cr\hline
    \multicolumn{2}{|l|}{Для кнопки удаления значения по ключу}\cr\hline
    \verb"(Name)" & \verb"RemoveBtn"\cr\hline
    \verb"Text" & \verb"Добавить"\cr\hline
    \multicolumn{2}{|l|}{Для текстового поля результата}\cr\hline
    \verb"(Name)" & \verb"ResultTextBox"\cr\hline
    \multicolumn{2}{|l|}{Для радиокнопки ключа по профессии}\cr\hline
    \verb"(Name)" & \verb"GetNamesBtn"\cr\hline
    \verb"Text" & \verb"Получить ключи по профессии"\cr\hline
    \multicolumn{2}{|l|}{Для радиокнопки профессии по ключу }\cr\hline
    \verb"(Name)" & \verb"GetPositionBtn"\cr\hline
    \verb"Text" & \verb"Получить профессию по ключу"\cr\hline
    \multicolumn{2}{|l|}{Для кнопки получения результата}\cr\hline
    \verb"(Name)" & \verb"ResultBtn"\cr\hline
    \verb"Text" & \verb"Получить результат"\cr\hline
    \multicolumn{2}{|l|}{Для кнопки сброса}\cr\hline
    \verb"(Name)" & \verb"ResetBtn"\cr\hline
    \verb"Text" & \verb"Сбросить фильтр"\cr\hline
    \multicolumn{2}{|l|}{Для обработчика ошибок}\cr\hline
    \verb"(Name)" & \verb"errorProvider1"\cr\hline

    \caption{Значения атрибутов элементов в приложении <<Матричный калькулятор}
    \label{table:params7}
\end{longtable}

Программа написана с использованием контейнеров из .NET framework и
работы с ними соответственно назначению кнопки. Ниже приведен пример работы
функции добавления элемента по ключу:

\begin{minted}[fontsize=\small, breaklines=true, style=bw, linenos]{cpp}
    private: System::Void AddBtn_Click(System::Object^ sender, System::EventArgs^ e) { // Добавление пары ключ - значение
			String^ name = NameTextBox->Text->ToString();
			String^ position = PosTextBox->Text->ToString();
			if (name == String::Empty || position == String::Empty) { // Если что-то не ввели - сообщим об этом
				errorProvider1->SetError(AddBtn, "Недопустимые значения");
				return;
			}
			if (!d.ContainsKey(name)) { // Если ключа нет, добавим
				d.Add(name, position);
				System::Collections::Generic::List<String^>^ lst = gcnew System::Collections::Generic::List<String^>(); // Создадим новый объект
				if (!p.ContainsKey(position)) 
					p.Add(position, lst);
				p[position]->Add(name); // Добавим в лист по этому ключу новое значение
				gridResult->Rows->Add(1);
				int _row = gridResult->RowCount;
				auto ResultPair = System::Collections::Generic::KeyValuePair<String^, String^>(name, position);
				FillRowWithDict(gridResult->Rows[_row - 1], ResultPair);
			}	
		}
\end{minted}

После запуска приложения открывается следующее окно:
\begin{figure}[H]
    \centering
    \includegraphics[scale=0.4]{task7/result3.png}
    \caption{Состояние приложения после добавление нескольких элементов по ключу}
\end{figure}

Ниже приведен пример работы программы:
После запуска приложения открывается следующее окно:
\begin{figure}[H]
    \centering
    \includegraphics[scale=0.4]{task7/result4.png}
    \caption{Результат поиска по значению}
\end{figure}

\begin{figure}[H]
    \centering
    \includegraphics[scale=0.4]{task7/result5.png}
    \caption{Результат поиска по ключу}
\end{figure}

В случае, если соответствующие поля не заполнены, выполнение программы
игнорируется.
Полный код программы приведен в приложении А.


%\section{Файловые диалоги и работа с файлами}

\textbf{Задание:} Создать таблицу Work. В другой файл вывести данные о рабочих, занимающих данную должность. (Вариант 14)

Вид окна представлен на рисунке \ref{fig:task8_form}.
\begin{figure}[H]
    \centering
    \includegraphics[scale=0.7]{task8/form.png}
    \caption{Внешний вид формы программы}
    \label{fig:task8_form}
\end{figure}

У элементов изменены значения некоторых атрибутов. 
Значения измененных атрибутов представлены в таблице \ref{table:params8}.

\begin{longtable}{|l|l|}
    Наименование атрибута & Значение\cr\hline
    \multicolumn{2}{|l|}{Для формы}\cr\hline
    \verb"Text" & \verb"Браузер "Амиго""\cr\hline
    \verb"FormBorderStyle" & \verb"FixedSingle"\cr\hline
    \verb"MaximizeBox" & \verb"False"\cr\hline
    \multicolumn{2}{|l|}{Для таблицы}\cr\hline
    \verb"(Name)" & \verb"gridResult"\cr\hline
    \multicolumn{2}{|l|}{Для кнопки записи в файл}\cr\hline
    \verb"(Name)" & \verb"SaveFileBtn"\cr\hline
    \multicolumn{2}{|l|}{Для кнопки открытия файла}\cr\hline
    \verb"(Name)" & \verb"OpenFileBtn"\cr\hline
    \multicolumn{2}{|l|}{Для кнопки сохранения результата}\cr\hline
    \verb"(Name)" & \verb"ResultBtn"\cr\hline
    \multicolumn{2}{|l|}{Для текстового поля должности}\cr\hline
    \verb"(Name)" & \verb"ResultTextBox"\cr\hline
    \multicolumn{2}{|l|}{Для обработчика ошибок}\cr\hline
    \verb"(Name)" & \verb"errorProvider1"\cr\hline

    \caption{Значения атрибутов элементов в приложении <<Работа с файлами>>}
    \label{table:params8}
\end{longtable}

Кроме того, приложение содержит элементы \verb|openFileDialogue|
и \verb|saveFileDialogue|, реализующие открытие и сохранение файлов.

Работа с ними производится в кнопках. Ниже приведен пример работы
функции открытия файла:

\begin{minted}[fontsize=\small, breaklines=true, style=bw, linenos]{cpp}
    private: System::Void OpenFileBtn_Click(System::Object^ sender, System::EventArgs^ e) {
			System::IO::Stream^ myStream;
			if (this->openFileDialog->ShowDialog() == System::Windows::Forms::DialogResult::OK) {
				CreateEmptyMatrix(gridResult, 1);
				if ((myStream = openFileDialog->OpenFile()) != nullptr) {
					System::IO::StreamReader^ sw = gcnew System::IO::StreamReader(myStream, System::Text::Encoding::
						GetEncoding(65001)); // UTF-8
					System::String^ s = "";
					int innerIdx = 0;
					while ((s = sw->ReadLine()) != nullptr && s != "") {
						gridResult->Rows->Add(1);
						int idx = 0;
						int current = 0;
						while (idx != s->Length) { 
							System::String^ currentWord = "";
							while (idx < s->Length && s[idx] != ' ') {
								currentWord += s[idx++];
							}
							if (idx < s->Length && s[idx] == ' ') ++idx;
							gridResult->Rows[innerIdx]->Cells[current++]->Value = currentWord;
							currentWord = "";
						}
						++innerIdx;
					}
					sw->Close();
				}
			}
		}
\end{minted}

После запуска программы на экране появляется окно следующего вида:
\begin{figure}[H]
    \centering
    \includegraphics[scale=0.4]{task8/exec.png}
    \caption{Внешний вид окна приложения}
\end{figure}

После открытия файла состояние программы изменится на следующее:
\begin{figure}[H]
    \centering
    \includegraphics[scale=0.4]{task8/openFile.png}
    \caption{Состояние программы после открытия файла}
\end{figure}

Результатом работы программы является новый файл:
\begin{figure}[H]
    \centering
    \includegraphics[scale=0.4]{task8/result.png}
    \caption{Результат работы программы}
\end{figure}

Полный код программы приведен в приложении А.


%\section{Приложение <<Тест>>}

\textbf{Задание:} Создать приложение для проведения тестирования. Оно должно содержать:

\begin{enumerate}
    \item Набор вопросов по какой-то теме (и вопросы и ответы должны быть реальные) -не менее 10
    \item Вопросы должны выбираться случайным образом.
    \item Вопросы должны быть нескольких типов - "Да/нет", Выбор одного ответа, Выбор нескольких ответов, Короткий ответ.
    \item Необходимо создать сообщения для правильного и неправильного ответа (Молодец, Не правильно и т.д.)
    \item Необходимо подсчитать количество правильных ответов и вывести результат.
\end{enumerate}

Приложение состоит из двух форм:

Вид окна представлен на изображениях \ref{fig:task9_form1}, \ref{fig:task9_form2}.
\begin{figure}[H]
    \centering
    \includegraphics[scale=0.6]{task9/form1.png}
    \caption{Внешний вид формы 1}
    \label{fig:task9_form1}
\end{figure}

\begin{figure}[H]
    \centering
    \includegraphics[scale=0.8]{task9/form2.png}
    \caption{Внешний вид формы 2}
    \label{fig:task9_form2}
\end{figure}


У элементов изменены значения некоторых атрибутов. 
Значения измененных атрибутов представлены в таблице \ref{table:params9}.

\begin{longtable}{|l|l|}
    Наименование атрибута & Значение\cr\hline
    \multicolumn{2}{|l|}{Для формы}\cr\hline
    \verb"Text" & \verb"Браузер "Амиго""\cr\hline
    \verb"FormBorderStyle" & \verb"FixedSingle"\cr\hline
    \verb"MaximizeBox" & \verb"False"\cr\hline
    \multicolumn{2}{|l|}{Для кнопки варианта 1}\cr\hline
    \verb"(Name)" & \verb"Option1Btn"\cr\hline
    \verb"Text" & \verb"Вариант 1"\cr\hline
    \multicolumn{2}{|l|}{Для кнопки варианта 2}\cr\hline
    \verb"(Name)" & \verb"Option2Btn"\cr\hline
    \verb"Text" & \verb"Вариант 2"\cr\hline
    \multicolumn{2}{|l|}{Для кнопки варианта 3}\cr\hline
    \verb"(Name)" & \verb"Option3Btn"\cr\hline
    \verb"Text" & \verb"Вариант 3"\cr\hline
    \multicolumn{2}{|l|}{Для кнопки варианта 4}\cr\hline
    \verb"(Name)" & \verb"Option4Btn"\cr\hline
    \verb"Text" & \verb"Вариант 4"\cr\hline
    \multicolumn{2}{|l|}{Для кнопки "Да"}\cr\hline
    \verb"(Name)" & \verb"YesBtn"\cr\hline
    \verb"Text" & \verb"Да"\cr\hline
    \multicolumn{2}{|l|}{Для кнопки "Нет"}\cr\hline
    \verb"(Name)" & \verb"NoBtn"\cr\hline
    \verb"Text" & \verb"Нет"\cr\hline
    \caption{Значения атрибутов элементов в приложении <<Приложение <<Тест>> >>}
    \label{table:params9}
\end{longtable}
После запуска приложения появляется окно следующего вида:

\begin{figure}[H]
    \includegraphics[scale=0.6]{task9/exec.png}
    \caption{Внешний вид окна после запуска приложения}
\end{figure}
Приложение содержит различные виды вопросов. Например, окно для вопросов с двумя вариантами
ответа выглядит следующим образом:

\begin{figure}[H]
    \includegraphics[scale=0.6]{task9/yesorno.png}
    \caption{Внешний вид для вопросов с двумя вариантами ответа}
\end{figure}

В коде приложения была создана структура наследовования, которая
упрощает работу с различными вариантами вопросов:

\begin{minted}[fontsize=\small, breaklines=true, style=bw, linenos]{cpp}
    ref struct Question {
		public: int questionId;
		public: Question() {};
		public: Question(int questionId, String^ text) : questionId(questionId), text(text) {};
		public: String^ text;
		public: virtual bool CheckAnswer() = 0;
		public: virtual Question^ Clone() = 0;
		};

		ref struct ShortAnswerQuestion : Question {
		public: String^ expectedAnswer;
		public: String^ userAnswer;
		public: ShortAnswerQuestion() {};
		public: ShortAnswerQuestion(int questionId, String^ expectedAnswer) {
			this->questionId = questionId;
			this->expectedAnswer = expectedAnswer;
		}
		public: virtual bool CheckAnswer() override {
			return expectedAnswer->Equals(userAnswer);
		}
		public: virtual Question^ Clone() override {
			ShortAnswerQuestion^ obj = (gcnew ShortAnswerQuestion());
			obj->expectedAnswer = this->expectedAnswer;
			obj->questionId = this->questionId;
			obj->text = this->text;
			obj->userAnswer = this->userAnswer;
			Question^ toBeReturned = obj;
			return toBeReturned;
		}
		};

		ref struct SeveralAnswerQuestion : Question {
		public: int count;
		public: int userAnswerId;
		public: int expectedAnswerId;
		public: virtual bool CheckAnswer() override {
			return userAnswerId == expectedAnswerId;
		}
		public: SeveralAnswerQuestion() {};
		public: SeveralAnswerQuestion(int questionId, int expectedAnswer, int count) {
			this->questionId = questionId;
			this->expectedAnswerId = expectedAnswerId;
			this->count = count;
		}
		public: virtual Question^ Clone() override {
			SeveralAnswerQuestion^ obj = (gcnew SeveralAnswerQuestion());
			obj->expectedAnswerId = this->expectedAnswerId;
			obj->questionId = this->questionId;
			obj->text = this->text;
			obj->count = this->count;
			Question^ toBeReturned = obj;
			return toBeReturned;
		}
		};
\end{minted}

В ходе выполнения программа сообщает пользователю о том, ввел 
ли он правильный ответ или нет:

\begin{figure}[H]
    \centering
    \includegraphics[scale=0.6]{task9/whoops.png}
    \caption{Окно в случае неправильного ответа на вопрос}
\end{figure}

\begin{figure}[H]
    \centering
    \includegraphics[scale=0.6]{task9/result.png}
    \caption{Окно в случае правильного ответа на вопрос}
\end{figure}
Программа не содержит исключительных ситуаций.


Полный код программы приведен в приложении А.



\conclusion

В ходе прохождения практики были получены основы разработки приложений
Windows Forms. Были изучены особенности и основные иструменты
.NET framework.

\appendix

\section{Репозиторий Github, содержащий полный код программ}


\bibliographystyle{gost780uv}
\bibliography{thesis}









%Библиографический список, составленный вручную, без использования BibTeX
%
%\begin{thebibliography}{99}
%  \bibitem{Ione} Источник 1.
%  \bibitem{Itwo} Источник 2
%\end{thebibliography}

%Библиографический список, составленный с помощью BibTeX
%
\bibliographystyle{gost780uv}
\bibliography{thesis}

\cite{MSDN}

% % При использовании biblatex вместо bibtex
%\printbibliography

\conclusion

% Окончание основного документа и начало приложений
% Каждая последующая секция документа будет являться приложением
\appendix

\end{document}
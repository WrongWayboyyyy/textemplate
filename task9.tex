\section{Приложение <<Тест>>}

\textbf{Задание:} Создать приложение для проведения тестирования. Оно должно содержать:

\begin{enumerate}
    \item Набор вопросов по какой-то теме (и вопросы и ответы должны быть реальные) -не менее 10
    \item Вопросы должны выбираться случайным образом.
    \item Вопросы должны быть нескольких типов - "Да/нет", Выбор одного ответа, Выбор нескольких ответов, Короткий ответ.
    \item Необходимо создать сообщения для правильного и неправильного ответа (Молодец, Не правильно и т.д.)
    \item Необходимо подсчитать количество правильных ответов и вывести результат.
\end{enumerate}

Приложение состоит из двух форм:

Вид окна представлен на изображениях \ref{fig:task9_form1}, \ref{fig:task9_form2}.
\begin{figure}[H]
    \centering
    \includegraphics[scale=0.6]{task9/form1.png}
    \caption{Внешний вид формы 1}
    \label{fig:task9_form1}
\end{figure}

\begin{figure}[H]
    \centering
    \includegraphics[scale=0.8]{task9/form2.png}
    \caption{Внешний вид формы 2}
    \label{fig:task9_form2}
\end{figure}


У элементов изменены значения некоторых атрибутов. 
Значения измененных атрибутов представлены в таблице \ref{table:params9}.

\begin{longtable}{|l|l|}
    Наименование атрибута & Значение\cr\hline
    \multicolumn{2}{|l|}{Для формы}\cr\hline
    \verb"Text" & \verb"Браузер "Амиго""\cr\hline
    \verb"FormBorderStyle" & \verb"FixedSingle"\cr\hline
    \verb"MaximizeBox" & \verb"False"\cr\hline
    \multicolumn{2}{|l|}{Для кнопки варианта 1}\cr\hline
    \verb"(Name)" & \verb"Option1Btn"\cr\hline
    \verb"Text" & \verb"Вариант 1"\cr\hline
    \multicolumn{2}{|l|}{Для кнопки варианта 2}\cr\hline
    \verb"(Name)" & \verb"Option2Btn"\cr\hline
    \verb"Text" & \verb"Вариант 2"\cr\hline
    \multicolumn{2}{|l|}{Для кнопки варианта 3}\cr\hline
    \verb"(Name)" & \verb"Option3Btn"\cr\hline
    \verb"Text" & \verb"Вариант 3"\cr\hline
    \multicolumn{2}{|l|}{Для кнопки варианта 4}\cr\hline
    \verb"(Name)" & \verb"Option4Btn"\cr\hline
    \verb"Text" & \verb"Вариант 4"\cr\hline
    \multicolumn{2}{|l|}{Для кнопки "Да"}\cr\hline
    \verb"(Name)" & \verb"YesBtn"\cr\hline
    \verb"Text" & \verb"Да"\cr\hline
    \multicolumn{2}{|l|}{Для кнопки "Нет"}\cr\hline
    \verb"(Name)" & \verb"NoBtn"\cr\hline
    \verb"Text" & \verb"Нет"\cr\hline
    \caption{Значения атрибутов элементов в приложении <<Приложение <<Тест>> >>}
    \label{table:params9}
\end{longtable}
После запуска приложения появляется окно следующего вида:

\begin{figure}[H]
    \includegraphics[scale=0.6]{task9/exec.png}
    \caption{Внешний вид окна после запуска приложения}
\end{figure}
Приложение содержит различные виды вопросов. Например, окно для вопросов с двумя вариантами
ответа выглядит следующим образом:

\begin{figure}[H]
    \includegraphics[scale=0.6]{task9/yesorno.png}
    \caption{Внешний вид для вопросов с двумя вариантами ответа}
\end{figure}

В коде приложения была создана структура наследовования, которая
упрощает работу с различными вариантами вопросов:

\begin{minted}[fontsize=\small, breaklines=true, style=bw, linenos]{cpp}
    ref struct Question {
		public: int questionId;
		public: Question() {};
		public: Question(int questionId, String^ text) : questionId(questionId), text(text) {};
		public: String^ text;
		public: virtual bool CheckAnswer() = 0;
		public: virtual Question^ Clone() = 0;
		};

		ref struct ShortAnswerQuestion : Question {
		public: String^ expectedAnswer;
		public: String^ userAnswer;
		public: ShortAnswerQuestion() {};
		public: ShortAnswerQuestion(int questionId, String^ expectedAnswer) {
			this->questionId = questionId;
			this->expectedAnswer = expectedAnswer;
		}
		public: virtual bool CheckAnswer() override {
			return expectedAnswer->Equals(userAnswer);
		}
		public: virtual Question^ Clone() override {
			ShortAnswerQuestion^ obj = (gcnew ShortAnswerQuestion());
			obj->expectedAnswer = this->expectedAnswer;
			obj->questionId = this->questionId;
			obj->text = this->text;
			obj->userAnswer = this->userAnswer;
			Question^ toBeReturned = obj;
			return toBeReturned;
		}
		};

		ref struct SeveralAnswerQuestion : Question {
		public: int count;
		public: int userAnswerId;
		public: int expectedAnswerId;
		public: virtual bool CheckAnswer() override {
			return userAnswerId == expectedAnswerId;
		}
		public: SeveralAnswerQuestion() {};
		public: SeveralAnswerQuestion(int questionId, int expectedAnswer, int count) {
			this->questionId = questionId;
			this->expectedAnswerId = expectedAnswerId;
			this->count = count;
		}
		public: virtual Question^ Clone() override {
			SeveralAnswerQuestion^ obj = (gcnew SeveralAnswerQuestion());
			obj->expectedAnswerId = this->expectedAnswerId;
			obj->questionId = this->questionId;
			obj->text = this->text;
			obj->count = this->count;
			Question^ toBeReturned = obj;
			return toBeReturned;
		}
		};
\end{minted}

В ходе выполнения программа сообщает пользователю о том, ввел 
ли он правильный ответ или нет:

\begin{figure}[H]
    \centering
    \includegraphics[scale=0.6]{task9/whoops.png}
    \caption{Окно в случае неправильного ответа на вопрос}
\end{figure}

\begin{figure}[H]
    \centering
    \includegraphics[scale=0.6]{task9/result.png}
    \caption{Окно в случае правильного ответа на вопрос}
\end{figure}
Программа не содержит исключительных ситуаций.


Полный код программы приведен в приложении А.

